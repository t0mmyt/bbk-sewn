%% LyX 2.1.3 created this file.  For more info, see http://www.lyx.org/.
%% Do not edit unless you really know what you are doing.
\documentclass[english]{article}
\usepackage[T1]{fontenc}
\usepackage[latin9]{inputenc}
\usepackage[landscape,a4paper]{geometry}
\geometry{verbose,tmargin=1cm,bmargin=2cm,lmargin=1.5cm,rmargin=1.5cm}

\makeatletter

%%%%%%%%%%%%%%%%%%%%%%%%%%%%%% LyX specific LaTeX commands.
%% Because html converters don't know tabularnewline
\providecommand{\tabularnewline}{\\}

%%%%%%%%%%%%%%%%%%%%%%%%%%%%%% User specified LaTeX commands.
\author{Tom Taylor - 13015452}

\makeatother

\usepackage{babel}
\begin{document}

\title{Parsing and filtering of web log data and data analysis}

\maketitle

\section{Prescribed queries}


\subsection{The top ten files/pages requested}

The following table shows the most common requests received by the
server. The top request appears to be for a CSS library (or similar)
so should probably be disregarded. The homepage appears to be 3rd
in the list of the remaining requests, possibly suggesting that traffic
is being linked to externally more than people are hitting the homepage.

\begin{tabular}{l|l}
\textbf{n} & \textbf{path}\tabularnewline
\hline 
1324 & /library/conditionalstyle.asp\tabularnewline
957 & /uk/letters/letters.asp\tabularnewline
872 & /uk/home/Default.asp\tabularnewline
631 & /Default.asp\tabularnewline
511 & /uk/financialcentre/tax\_calculator\_tool.asp\tabularnewline
471 & /uk/letters/default.asp\tabularnewline
388 & /uk/letters/resignation\_letter\_generator\_form\_v2.asp\tabularnewline
387 & /uk/discussion/new\_topic.asp\tabularnewline
281 & /uk/letters/resignation\_letter\_generator\_generate\_v2.asp\tabularnewline
206 & /uk/financialcentre/tax\_calculator.asp\tabularnewline
\end{tabular}


\subsection{The top ten IP addresses (or users) who requested the most URLs.}

From the data provided, there was never any username set so the username
was disregarded in the query. The following shows the top 10 IPs.

\begin{tabular}{l|l}
\textbf{n} & \textbf{IP}\tabularnewline
\hline 
554 & 195.149.39.85\tabularnewline
328 & 65.214.36.156\tabularnewline
194 & 65.214.36.152\tabularnewline
192 & 213.199.149.236\tabularnewline
88 & 209.140.222.149\tabularnewline
84 & 195.92.168.177\tabularnewline
79 & 62.254.0.7\tabularnewline
72 & 192.168.1.6\tabularnewline
60 & 12.47.98.180\tabularnewline
58 & 62.255.64.5\tabularnewline
\end{tabular}


\subsection{The top three most active hours (most requests per hour).}

The following table shows the three busiest hours.

\begin{tabular}{l|l}
\textbf{n} & \textbf{hour}\tabularnewline
\hline 
645 & 20\tabularnewline
608 & 14\tabularnewline
597 & 18\tabularnewline
\end{tabular}


\subsection{The number of requests per query method.}

The majority of requests were GET requests which would be expected
from a WWW webserver. The 10 that had no method type were also errors
of various types.

\begin{tabular}{l|l}
\textbf{n} & \textbf{method}\tabularnewline
\hline 
9583 & GET\tabularnewline
240 & POST\tabularnewline
10 & HEAD\tabularnewline
10 & -\tabularnewline
\end{tabular}


\section{Additional queries}


\subsection{Errors served with internal referrals}

The following table shows error documents that were served from a
visitor following an internal link. This would be of interest to the
webmasters/content owners/developers to identify potential faults
with the content or application server.

\begin{tabular}{l|l|l|l}
\textbf{n} & \textbf{path} & \textbf{status} & \textbf{referrer}\tabularnewline
\hline 
6 & /uk/letters/letters/letterform.asp & 404 & http://www.i-resign.com/uk/letters/kissmyass\_resign.asp\tabularnewline
6 & /uk/letters/workinglife/viewarticle\_4.asp & 404 & http://www.i-resign.com/uk/letters/dilbert\_resign.asp\tabularnewline
3 & /uk/letters/halloffame/ & 404 & http://www.i-resign.com/uk/letters/kissmyass\_resign.asp\tabularnewline
3 & /us/financialcenter/\_tc.asp & 500 & http://www.i-resign.com/us/financialcenter/federal\_tax\_estimator\_2.asp\#form\tabularnewline
3 & /uk/letters/letters/http:/www.i-resign.com & 404 & http://www.i-resign.com/uk/letters/letters/letterform.asp\tabularnewline
2 & /uk/letters/resignation\_letter\_generator\_generate\_v2.asp & 500 & http://www.i-resign.com/uk/letters/resignation\_letter\_generator\_generate\_v2.asp?country=\&RT=move\&yourname=Marilou+Messina\&nperiod=2\&job\_title=Crder+Entr\%2FCustoer+Service\&department\_name=\&bossname=Mr.+Michael+T.+Gill\%2C+Controler\&companyname=J\%26M+Reproduction\&firmaddress=1200+Rochester+Road\&firmcity=Troy\&firmregion=MI\&post\_code=48083\tabularnewline
2 & /uk/stress/ & 404 & http://www.i-resign.com/uk/legaladvice/top.asp\tabularnewline
2 & /uk/letters/letters/http:/http:/www.i-resign.com & 404 & http://www.i-resign.com/uk/letters/letters/http\%3A//www.i-resign.com\tabularnewline
1 & /cgi-bin/formmail.pl & 404 & http://www.i-resign.com/\tabularnewline
1 & /cgi-bin/formmail.cgi & 404 & http://www.i-resign.com/\tabularnewline
1 & /uk/discussion/new\_topic.asp & 500 & http://www.i-resign.co.uk/uk/search/Default.asp?free=\&query=holiday+pay\&p=1\tabularnewline
\end{tabular}

\pagebreak{}


\subsection{Traffic from Search Engines}

In Januray 2003, according to OneStat, the top three search engines
were Google, Yahoo and MSN with 54.7\%, 22.1\% and 9.5\% respectively.
The following table shows the number of requests from each of those
engines. They are not shown as a \% as that would be misleading (not
all search engines were considered so the traffic would be skewed)
but it can be seen that they are similar in ratio to the reported
statistics of the time.

\begin{tabular}{l|l}
\textbf{n} & \textbf{search\_engine}\tabularnewline
\hline 
407 & Google\tabularnewline
181 & MSN\tabularnewline
300 & Yahoo\tabularnewline
\end{tabular}


\section{Appendix}


\subsection{Queries that would be interesting but no longer relevent}

It would have been interesting to compare the IP addresses with Geo
location datasets to analyse where traffic is coming from. It would
also have been interesting to query PTR records for the IPs to more
accurately identify crawlers and bots and also get a snapshot of ISPs.
Unfortunately the current IP data would be far out of date now.
\end{document}
